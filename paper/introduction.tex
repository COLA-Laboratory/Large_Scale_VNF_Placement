\section{Introduction}
% Provide context
Modern communication services have and continue to have an outsized impact on the way that society works \cite{OECD16}, travels \cite{ZhengMF15}, socializes \cite{Bargh04} and engages with politics \cite{Farrell12}, government \cite{ChadwickM13} and other individuals \cite{Castells14}. But this does not come without a cost. These services are also major consumers of electricity, with communications technology as a whole expected to be responsible for 21\% of the worlds total electricity usage by 2030 \cite{AndraeE15}.

If current trends continue, by 2030 data centers will likely be the largest single contributor of carbon emissions among communications infrastructure \cite{AndraeE15}. Over the previous decade, extensive work was conducted to reduce the energy consumed by overhead, such as heat management and energy provisioning, in data centers \cite{AvgerinouBC17}. Through these efforts, the energy total consumption of data centers remained constant in the United States \cite{Shehabi18} and doubled in the European union \cite{DoddAGC20} from 2010 - 2020, despite a forecasted 10$\times$ increase in data center traffic \cite{Cisco18}. However, in recent years, progress in this direction has slowed, as improvements to overhead efficiency have reached the point of diminishing returns \cite{Google21}.

Recent research found that the efficiency of a data center could be improved by a further 10\% to 40\% by maximizing the utilization of existing computing equipment \cite{DoddAGC20,ShehabiARSSD16}. However, existing communication services make extensive use of specialized hardware which can only be utilized for a single task. Specifically, communication services are typically constructed using sequences of purpose built computing equipment known as middleboxes or physical network functions. Each middlebox can perform a single task with high efficiency, but being physical components, the number and location of these middleboxes must be specified well in advance of their usage. A recent alternative are Virtual Network Functions (VNFs). VNFs use software running on virtual machines to perform the same tasks as middleboxes. Using VNFs, multiple network functions can be provided by the same server, maximizing the utilization of hardware which results in less energy being needed to provide the same quality of service (QoS). In addition, the resources used by VNFs can be moved and scaled to meet traffic demands without over or under allocating resources.

Although VNFs allow increased utilization of servers, it remains an open problem how these VNFs should be assigned to servers. The VNF Placement Problem (VNFPP) is the task of automatically determining the placement of VNFs in a data center so as to provide high quality services with low energy consumption. The VNFPP presents many challenges. First, it is widely acknowledged that VNFPP is a NP hard problem~\cite{CohenLNR15,LuizelliCBG17,SangJGDY17} for which there are no known solutions that can find an exact solution in reasonable time. Second, whilst the QoS objectives and energy consumption are both well understood, they are challenging to model accurately and efficiently. Finally, due to the large numbers of servers and VNFs in a data center the total search space is very large whilst the feasible search space is relatively small \cite{BillingsleyLMMG22} which makes finding feasible solutions challenging, and high quality solutions even more so.

In prior work \cite{BillingsleyLMMG22}, we proposed a combined metaheuristic and model solution with a genotype-phenotype mapping that found good solutions to the VNFPP despite these challenges and which could solve 8x larger problems than the state of the art. However, the proposed algorithm was only able to solve problems for one type of network topology. More recently, we proposed a modification of our original algorithm that allowed the algorithm to work on arbitrary network topologies but this approach had limited scalability \cite{BillingsleyLMMG20}. In this paper we build on our earlier work to make further improvements and resolve the scalability issues. In particular, this paper combines our generalized genotype-phenotype mapping with the following innovations:

\begin{enumerate}
    \item A new parallel metaheuristic that uses decomposition and local search to efficiently search the large solution space.
    \item Efficient objective functions that balance model accuracy against evaluation time.
    \item Memory efficient data structures that significantly reduce the memory consumption of the metaheuristic
\end{enumerate}

Using these innovations, our solution is the first to solve the VNFPP for $\sim$65,000 servers. This represents a further 4x improvement over our earlier solution, and a 16x improvement over the preceding state of the art.

% Paragraph to lead into the paper
The remainder of this paper is organized as follows. Section \ref{sec:literature_review} reviews the current work on the VNFPP. Section \ref{sec:problem_formulation} provides a formal definition of the VNFPP. Section \ref{sec:optimisation} describes the parallel metaheuristic and operators considered in this work including the memory efficient data structures. Section \ref{sec:practical_objective_functions} discusses common objective functions for the VNFPP. In Section \ref{sec:evaluation} we test the effectiveness of each component of the algorithm and evaluate our solution on very large problem instances. Finally, Section \ref{sec:conclusion} concludes this paper and outlines some potential future directions.
