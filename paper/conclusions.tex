\section{Conclusions}
\label{sec:conclusion}
In this paper we proposed a new MOEA for arbitrary graphs capable of efficiently solving large scale instances of the VNFPP. To this end we proposed:
\begin{itemize}
    \item A novel solution representation for large scale, arbitrary network topologies.
    \item A novel local search based, parallel MOEA that significantly reduces the total algorithm execution time whilst achieving greater solution diversity.
    \item An improved initialization algorithm that ensures high quality, diverse solutions.
    \item A novel heuristic model that captures sufficient information about the VNFPP whilst greatly reducing the evaluation time.
\end{itemize}
We evaluated the effectiveness of each of these components and found they made significant improvements over state-of-the-art alternatives. By synthesizing these components we demonstrated that our algorithm could solve problem instances with up to 64,000 servers, a 16$\times$ improvement over problems that have been considered in the literature so far. Further, we found that the improved initialization operator combined with the genotype-phenotype solution representation is highly effective on very large problem instances.

Further extensions could be considered in future work. 
\begin{itemize}
    \item Our proposed operators combined produce an effective heuristic for the multi-objective VNFPP. It would be interesting to apply the same principles to the single-objective VNFPP which aim to minimize a cost metric and treat QoS requirements as a constraint.
    \item A challenging problem that remains unresolved is \textit{service resilience} i.e. the capacity for the data center to continue to provide a service when components may fail. This is an instance of a well known NP-Hard problem, graph partitioning, hence approximation algorithms and heuristics will likely be required.
    \item Finally, future work could extend our metaheuristic into a dynamic optimization problem such as in \cite{OtokuraLKKSM16}. Metaheuristics are well suited to dynamic optimization problems and many algorithmic frameworks have been proposed \cite{AlbaNS13}. This would enable our algorithm to adapt to changing requirements and to present the best set of options at any given moment.
\end{itemize}